\chapter{Visualization Methods}
\label{Chapter3}
\lhead{Chapter 3. \emph{Visualization Methods}}

mathematics

This section attempts to discuss some theory about viewing fields of surveillance cameras and methods to visualize them.

The term viewing field of camera is used to describe the angular extent of a given scene that is imaged by that camera, as illustrated in Figure [cite].

In 3D geometry, viewing field of a camera is defined by an infinite half cone, apex of which corresponds to the focal point of the camera. However, to visualize the cone in practical way, we limit the top and bottom of the cone and express it by a frustum as illustrated in the Figure [cite] and [cite]. In fact the viewing field is the volume bounded by the frustum and the 3D model.

We let the user manually select a visualization that best suits his preference, or even let him combine the methods. For example, Figure [cite] is actually a combination of the silhouette method and the arrow method. Experiments in section 4 will attempt to visually and practically find and verify the best understandable methods or any combinations of them.

\section{Volume method}

As discussed above, the natural method to visualize the viewing field is thus simply visualizing this volume as shown in Figure [cite]. We call this method "volume (visualization) method". However, the volume usually occupies all the scene viewed from the mobile camera, especially if the user is inside the viewing volume. The visualization produced by this method is hard to understand.

\section{Shadow method}

The fact that the image captured by the camera is perspective projection of the volume into the near plane of the frustum gives us another method. This method visualizes the virtual shadow created by the light positioned at the eye-point of the camera and the near plane of the frustum, as shown in Figure [cite]. We call this method "shadow (visualization) method". This method gives more understandable visualization. There are various ways to render the shadow in Figure [cite], the two popular ones are shadow mapping \citep{Reference7} \citep{Reference8} and volume shadow \citep{Reference9}.

\section{Silhouette method}

An alternative method is to only visualize the silhouettes of the shadow, as shown in Figure [cite]. We call this method "silhouette (visualization) method". Rendering only the silhouette is much lighter than rendering the whole shadow, thus this method is supposed to be much faster.

\section{Arrow method}

For most of the time, a person may want to know not only the viewing field, but also the position and distance to the surveillance camera, To visualize this information, we propose a method using vectors as shown in Figure [cite]. In the figure, all the vectors are pointing the camera, and the length of the vectors are reverse proportional to the distance from the root of the vectors to the camera. We call this method "arrow (visualization) method".

\section{Animation method}

We are studying the visualization method in which animate objects coming from the surveillance cameras are used to visualize the viewing fields. This method has  many advantages:

\begin{itemize}
\item The moving objects start form the surveillance camera positions, as a result the user easily knows the camera positions.
\item Event when the mobile camera is fixed at a certain position and orientation, the visualization effect is still achieved.
\item The moving objects do not block the scene, as a result the user can see the scene and visualized camera viewing fields at the same time.
\end{itemize}
