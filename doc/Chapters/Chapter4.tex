\chapter{Evaluation of Visualization Methods}
\label{Chapter4}
\lhead{Chapter 4. \emph{Evaluation of Visualization Methods}}

In this chapter, we use the prototype implemented in chapter \ref{Chapter3} to conduct an experiment to evaluate the five visualization methods in section \ref{VisualizationMethods}. Another purpose is to verify the requirements in section \ref{VisualizationRequirements}.

%------------------------------------------------------------------------------

\section{Evaluation Condition}

Ten users of about 20-30 years old are invited to use the system with the five visualization methods. After that, each user are asked to rank each visualization method in the following categories in five levels:

\begin{itemize}
	\item Understandability
	\item Realtimeness
	\item Accuracy
	\item Robustness to the disturbance of passers-by and handshaking movement of the mobile device
\end{itemize}

3D model:The system does not use a server or any real surveillance camera. In place of those, 3D model of the building of the College of International Studies and the gymnasium (Figure [cite]), together with the hard coded geometry information about the virtual surveillance cameras are used.

%------------------------------------------------------------------------------

\section{Result and Discussion}

%A good visualization method is the one in which CG objects used to visualize the viewing fields do not occlude the original scene. Moving objects do not occlude the scene over time, thus
