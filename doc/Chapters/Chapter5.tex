\chapter{Conclusion and Future Directions}
\label{Chapter5}
\lhead{Chapter 5. \emph{Conclusion and Future Directions}}

% In this research, we have proposed an outdoor MR system which enables a user to easily use his mobile device to have a visualization of viewing fields of surveillance cameras surrounding himself in realtime.
% 
% * server side: multicore
% * client side:
% 	* Actually, processing work on the VAIO is light (send image, display ...)=> tweak to run on lower spec mobile devices like Apple iPhone, Google T-Mobile G1.
% 	* 5 years
% * multi-sensor fusion
% 
% 
% In the contrary, at this time the above devices are limited in memory and computing power because of their small size, thus usually not very suitable for MR applications (and image processing applications, in general), which usually need large memory and powerful computing capability.
% 
% => 5 years
% 
% thuong co san ca gps va accelerator => sensor fusion
% 
% Moore's Law is the observation that the amount you can do on a single chip doubles every two years. But Moore's Law is taking a detour. Rather than producing faster and faster processors, companies such as Intel and AMD are producing multi-core devices: single chips containing two, four, or more processors. If your programs aren't concurrent, they'll only run on a single processor at a time. Your users will think that your code is slow.
% 
% Erlang is a programming language designed for building highly parallel, distributed, fault-tolerant systems. It has been used commercially for many years to build massive fault-tolerated systems that run for years with minimal failures.
% 
% Erlang programs run seamlessly on multi-core computers: this means your Erlang program should run a lot faster on a 4 core processor than on a single core processor, all without you having to change a line of code.
% 
% Erlang combines ideas from the world of functional programming with techniques for building fault-tolerant systems to make a powerful language for building the massively parallel, networked applications of the future.
% 
% This book presents Erlang and functional programming in the familiar Pragmatic style. And it's written by Joe Armstrong, one of the creators of Erlang.
% 
% 
% Future directions:
% * Co 2 huong: 1 la xu li tren 1 cai luon k can server => (One-)hand held mobile device co hardware spec ngay cang lon.
% * 1 la dung server, nhung communication network phai nhanh => Cai iPhone gi do (nho cite) co the len den 45FPS! Mang cung nhanh <---
% 
% Moore's laws bi limit roi => vs. Tien the noi ve xu huong multicore CPU => van de server k phai lo nhieu!
% Nho cite nhieu nhieu, cai thesis Erlang cua ong gi nhi?

iPhone: ping 2ms
  k compress: 5.27
