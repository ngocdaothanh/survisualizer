\chapter{Conclusion}
\label{Chapter5}

This research aims to visualize the viewing fields of outdoor surveillance cameras on the screen of mobile devices. In this research, we have proposed:

\begin{itemize}
	\item The Volume, shadow, contour, arrow and animation methods to visualize viewing fields of surveillance cameras for outdoor MR.
	\item A new architecture of online camera registration: server-side PTAM \cite{Reference12} system. Our prototype system can provide realtime MR video with no requirement for any GPS or gyrocompass devices, although we can also take these sensors to make the system more reliable.
\end{itemize}

We have also conducted the evaluation experiment on the visualization methods, using the prototype system. The experiment shows that the prototype system is usable even with a device with limited capability like the iPhone, and the server-side PTAM architecture is practical for outdoor MR.

After pushing heavy work to the server side, the client side has less work left: sending video images and displaying visual aids. We could use the iPhone which is a cell phone as the mobile devices in the prototype system, rather than UMPCs like the VAIO. Many existing games on high-end cell phones have proved that they are able to display 3D CG perfectly using OpenGL ES. Because cell phone is a big market, if we can push outdoor MR to cell phones, the market will push back and become a great force for the advance of outdoor MR. Moreover, today's cell phones are usually equipped with GPS and accelerometer devices. We can use them together with the camera to create a multisensor fusion \cite{Reference14} to improve the accuracy of the system.

Moore's Law is the observation that the amount you can do on a single chip doubles every two years. But it may have gradually come to an end: devices can hardly get smaller and faster. We cannot only sit and hope that mobile devices will constantly become much faster without changing their physical sizes. We should find other options. A new trend in computer processor technology is multicore. Rather than producing faster and faster CPUs, companies such as Intel and AMD are producing multicore ones: single chips containing two, four, or more cores. Within several years there may be hundreds of cores in a CPU. Each PTAM session requires two cores. The server in our prototype system is a dualcore and can only process one PTAM session at a time. A server with more cores can simultaneously process more PTAM sessions.

Multicore hardware requires parallel processing software. The frequency of a CPU does not go higher (about 3--4 GHz as of current time), but its computation potential goes higher because there are many cores integrated into it. If a program is not parallel, it will only run on a single core at a time and users will think that the program is slow. However, parallel programming is hard and must be further studied \cite{Reference17}.

With these technological advances, we believe that in the future more powerful viewing field visualization systems can be built and put in wide use, and they can help realize more safe society.
