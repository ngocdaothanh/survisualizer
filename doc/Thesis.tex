\documentclass[a4paper,11pt]{report}

%%[inserting images of PostScript, JPEG, PNG and so on]
%% choose and comment-out the following packages
\usepackage{graphicx} % for \includegraphics[width=3cm]{sample.eps}
%\usepackage{epsfig} % for \psfig{file=sample.eps,width=3cm}
%%\usepackage{epsf} % for \epsfile{file=sample.eps,scale=0.6}
%\usepackage{epsbox} % for \epsfile{file=sample.eps,scale=0.6}

%% You may use dvipdfm for directly creating PDF from DVI.
%\usepackage[dvipdfm]{color,graphicx}
%% bookmarking with dvipdfm
%\usepackage[dvipdfm,bookmarks=true,bookmarksnumbered=true,bookmarkstype=toc]{hyperref}

%\usepackage{times} % use Times Font instead of Computer Modern

\setcounter{tocdepth}{3}
\setcounter{page}{-1}

\setlength{\oddsidemargin}{0.1in}
\setlength{\evensidemargin}{0.1in} 
\setlength{\topmargin}{0in}
\setlength{\textwidth}{6in} 
%\setlength{\textheight}{10.1in}
\setlength{\parskip}{0em}
\setlength{\topsep}{0em}

%\newcommand{\fig}[1]{{\bf Fig.\ref{#1}}}

%% [IMPORTANT] package for title generation (english version)
\usepackage{sie-en}

%% title of thesis
%% DON'T PUT \\ AT THE END OF THE TITLE. IT CAUSES ERROR!!
\title{Visualization of Viewing Fields of Surveillance Cameras by Using Mixed Reality}
%% name of author
\author{Dao Ngoc Thanh}
%% name of advisor
\advisor{Yuichi Ohta}

%% major and degree and date (chooose one)
%% [NOTICE] Month varies with majors. (submitted date)
%\majorfield{Policy and Planning Science} \degree{Economics} \yearandmonth{January 2002}
%\majorfield{Policy and Planning Science} \degree{Policy and Planning Sciences} \yearandmonth{January 2002}
%\majorfield{Policy and Planning Science} \degree{Engineering} \yearandmonth{January 2002}
%\majorfield{Quantitative Finance and Management} \degree{Finance} \yearandmonth{January 2002}
%\majorfield{Quantitative Finance and Management} \degree{Management} \yearandmonth{January 2002}
%\majorfield{Quantitative Finance and Management} \degree{Policy and Planning Sciences} \yearandmonth{January 2002}
%\majorfield{Computer Science} \degree{Engineering} \yearandmonth{February 2002}
%\majorfield{Advanced Engineering Systems} \degree{Engineering} \yearandmonth{February 2002}
%\majorfield{Engineering Mechanics and Energy} \degree{Engineering} \yearandmonth{February 2002}
%\majorfield{Risk Engineering} \degree{Engineering} \yearandmonth{2002}
%\majorfield{Risk Engineering} \degree{Policy and Planning Sciences} \yearandmonth{2002}

\majorfield{Advanced Engineering Systems} \degree{Engineering} \yearandmonth{January 2009}

\begin{document}
\maketitle
\thispagestyle{empty}
\newpage

\thispagestyle{empty}
\vspace*{20pt plus 1fil}
%\parindent=1zw
\noindent
%%
%% Abstract
%%
\begin{center}
{\bf Abstract}
\vspace{5mm}
\end{center}
This research lies in the domain of Outdoor Mixed Reality. It aims to visualize the viewing fields of outdoor surveillance cameras on the screen of mobile devices. In an outdoor environment where cameras on mobile devices can be used, it is required that users can easily understands the visualized viewing fields of surrounding surveillance cameras. Moreover, the visualization system must provide realtime video with no human-sensible delay.

In this research, we study five visualization methods and propose a prototype implementation based on server-side PTAM. The prototype requires no GPS or gyrocompass devices.

Structure of this thesis:

\begin{itemize}
	\item Chapter \ref{Chapter1} introduces the need of visualizing viewing fields of surveillance cameras, and a background of related technologies and researches.
	\item Chapter \ref{Chapter2} poses basic requirements for visualization methods, then studies five methods and algorithms to implement them.
	\item Chapter \ref{Chapter3} propose a prototype based on server-side PTAM which implements and integrates the five methods in chapter \ref{Chapter2}.
	\item Chapter \ref{Chapter4} uses the prototype to conduct an experiment to evaluate the five methods in the context of the prototype, then discusses the result.
	\item Chapter \ref{Chapter5} summarizes this research and proposes future directions.
\end{itemize} 

%%%%%
\par
\vspace{0pt plus 1fil}
\newpage

\pagenumbering{roman} % I, II, III, IV 
\tableofcontents
\listoffigures
%\listoftables

\pagebreak \setcounter{page}{1}
\pagenumbering{arabic} % 1,2,3

\chapter{Introduction}
\label{Chapter1}

%------------------------------------------------------------------------------

\section{Introduction}

Surveillance cameras have become popular and placed almost everywhere, e.g. train stations, airports, banks, shops, streets. The balance between security and privacy is an active subject for political/social debates. It is hard to keep a balance between security and privacy. A person may feel protected when he knows himself and his surroundings are being observed, but at the same time he may feel uncomfortable.

The information that people probably wants to know most about surveillance cameras is probably their viewing fields, e.g. where the cameras are, whether he is inside the area being observed. On the other hand, for indoor environment, because the space is usually small, people may easily locate the position and estimate the viewing fields of the cameras. For outdoor environment, it is rather hard to find surveillance camera systems hence they need special support to visually know the viewing fields in real scene. Fortunately, there is a technology which can be applied for this demand: Mixed Reality (MR). MR is encompassing both Augmented Reality and Augmented Virtuality, merging the real world in which we are living and the virtual world created by computers. MR can produce new environments where real and virtual entities can co-exist and interact in realtime \cite{Reference3}.

MR applications are traditionally equipped with Head Mounted Displays (HMD) as shown in figure \ref{fig:HMD}. However, HMDs are usually bulky and inconvenient for outdoor applications, where high-level mobility is crucial. In recent years, mobile devices have become popular, devices like cell phones, Personal Digital Assistants (PDA) can be seen everywhere. Their prices have come down and they could reach the hands of ordinary people even in developing countries like Viet Nam. High-end mobile devices, such as Apple iPhone and Google T-Mobile G1, usually have high specifications. As for their important advantages for outdoor MR applications, they are almost always equipped with built-in cameras, Global Positioning System (GPS), and accelerometer devices. These auxiliary devices usually meet the requirements on building up outdoor MR applications because they can provide video and information to help estimating position and orientation of the mobile devices \cite{Reference2} \cite{Reference4}.

\begin{figure}[htbp]
	\centering
	\includegraphics{./Primitives/hmd.jpg}
	\rule{35em}{0.5pt}
	\caption[Head Mounted Dislay]{A Head Mounted Dislay}
	\label{fig:HMD}
\end{figure}

GPS and accelerometer devices are not sufficient for achieving highly precise MR which we need to visualize viewing field of surveillance cameras. Hence we need to introduce an image-based camera registration method which could reach to the accuracy we need at frame rate. However, image processing applications usually require large memory and powerful computing capability. Even current high-end cell phones and PDAs may not run MR applications properly without special program optimizations. For computation-intensive outdoor MR applications, more powerful Ultra-Mobile Portable Computers (UMPC) like the one in figure \ref{fig:VAIO} may be used. UMPCs equipped with built-in camera, GPS, and gyrocompass devices have been found practical and are the topic of many researches \cite{Reference2} \cite{Reference4} \cite{Reference13}.

\begin{figure}[htbp]
	\centering
	\includegraphics[width=14cm]{./Primitives/vaio.png}
	\rule{35em}{0.5pt}
	\caption[Sony VAIO VGN-UX90PS]{Sony VAIO VGN-UX90PS with a camera at the front and at the back}
	\label{fig:VAIO}
\end{figure}

This research aims to visualize the viewing fields of outdoor surveillance cameras on the screen of mobile devices. In this research, we propose:

\begin{itemize}
	\item Five methods to visualize viewing fields of surveillance cameras for outdoor MR.
	\item A new architecture of online camera calibration: server-side Parallel Tracking and Mapping (PTAM) \cite{Reference12} system. Our prototype system uses a mobile device which can be either a Sony VAIO VGN-UX90PS or an Apple iPhone and can provide realtime video. Because the mobile devices do not have sufficient computation facility to execute the heavy PTAM processing, we connect it wirelessly to a PTAM server which is an Apple MacBook Pro. This results in a system providing realtime MR video with no requirement for any GPS or gyrocompass devices.
\end{itemize}

Structure of this thesis:

\begin{itemize}
	\item Rest of chapter \ref{Chapter1} introduces the requirement on visualizing viewing fields of surveillance cameras, and some related technologies and researches.
	\item Chapter \ref{Chapter2} poses basic elements of visualization methods, then studies five visualization methods to show viewing fields of surveillance cameras.
	\item Chapter \ref{Chapter3} proposes online camera registration method based on server-side PTAM which implements and is integrated to the visualization methods in chapter \ref{Chapter2}.
	\item Chapter \ref{Chapter4} conducts an experiment to evaluate the five methods using the prototype, then discusses the result.
	\item Chapter \ref{Chapter5} summarizes this research and proposes future directions.
\end{itemize}

%------------------------------------------------------------------------------

\section{Use Case}
\label{UseCase}

When a user wants to see the visualized viewing fields of surrounding surveillance cameras on his own mobile device screen, the typical usage scenario is (see figure \ref{fig:UseCase} and \ref{fig:PrototypeArchitecture}):

\begin{enumerate}
	\item The user points the mobile camera in a certain direction.
	\item Through wireless network, the mobile device continuously sends video frames captured by the camera of the scene to a PTAM server.
	\item The PTAM server estimates the position and orientation of the mobile camera by analyzing the received video frames, and send them to the mobile device.
	\item Based on the position and orientation, the mobile device will correctly render visual aids that are instances of viewing fields of the surveillance cameras, and shows the MR video images on its screen.
	\item When the user changes the position or orientation of the mobile camera, the MR video displayed on the screen will simultaneously change accordingly to the camera movement.
\end{enumerate}

\begin{figure}[htbp]
	\centering
	\includegraphics[width=14cm]{./Primitives/usecase.png}
	\rule{35em}{0.5pt}
	\caption[Usage scenario]{A usage scenario}
	\label{fig:UseCase}
\end{figure}

\begin{figure}[htbp]
	\centering
	\includegraphics[width=14cm]{./Figures/prototype_architecture.png}
	\rule{35em}{0.5pt}
	\caption[Prototype architecture]{Prototype architecture. The mobile device can be either a VAIO (appendix \ref{AppendixVAIO}) or an iPhone (appendix \ref{AppendixiPhone}).}
	\label{fig:PrototypeArchitecture}
\end{figure}

The above use case gives an example of the application of visualizing viewing fields of surveillance cameras and a rough idea of how the prototype works. There may be other applications. For example, we can build a system which visualizes ``safe paths'' or ``safe areas'' for children (figure \ref{fig:HomeSchool}). Children can use their cell phones to see if their current location is being well observed when they walk to school or back home.

\begin{figure}[htbp]
	\centering
	\includegraphics{./Primitives/home_school.png}
	\rule{35em}{0.5pt}
	\caption[Safe path to school]{Safe path to school}
	\label{fig:HomeSchool}
\end{figure}

%------------------------------------------------------------------------------

\section{Related Works}

For outdoor MR, bulky devices that reduce the mobility of users are not preferable. For example: heavy computers, devices wired to a fixed location, devices with big auxiliaries, big HMDs, HMDs that prevent users from seeing the outdoor environment while walking etc. Devices for outdoor MR should be small, lightweight, and can connect wirelessly to other devices if they need to.

MR systems usually need to know the position and orientation of the devices in order to merge images of the virtual world and the images of the real world at high accuracy. For indoor environment, marker-based solutions are known to work very well \cite{Reference20}. However, outdoor environment is usually large and it is impossible to put markers everywhere. There have been many researches that use UMPCs equipped with GPS and/or gyrocompass devices to realize markerless camera registration \cite{Reference2} \cite{Reference4} \cite{Reference13}. Hardware devices have been becoming smaller and more powerful according to Moore's law. Today's high-end cell phones, like Apple iPhone and Google T-Mobile G1, usually have built-in GPS and accelerometer devices. Because of their small size, high specifications, and competitive prices, such devices have found their use in outdoor MR applications and researches, like Sekai camera \cite{Reference18} and Enkin \cite{Reference19}.

However, normal GPS devices have error of about 5--10 m, and gyrocompass devices suffer from drift error. There have been many researches that are based on the images taken by the camera on the mobile device to deal with this problem. We can follow the approach that uses (1) gravity accelerometer to help eliminating the drift error, (2) model-based tracking on edges and (3) texture tracker to help produce accurate localization result \cite{Reference13}. There is also an approach that completely does not require GPS or gyrocompass devices, but uses the images taken by the camera and a landmark database of natural feature points built before-hand \cite{Reference21}.

When MR programs need more memory or CPU power than what the mobile devices can provide, we have a backup solution in which the mobile devices connect wirelessly to remote non-mobile computers to ask for help. It is possible to send realtime video images wirelessly to remote computers for further processing because 54 Mbps wireless network has become popular today. This may be a good approach because Moore's law has slowed down in recent years, thus we cannot expect the computation power of mobile devices to go much higher anytime soon while at the same time keeping their small size and mobility.

\chapter{Visualization Methods}
\label{Chapter2}

The purpose of our research is to visualize the subspace of the real world in which any objects could be inserted by surveillance cameras. We take up ordinary perspective cameras for surveillance cameras in this research. We propose to insert ``visual aids'' into real images captured by the mobile camera. We take up five kinds of visual aids in this paper. First we discuss the elements of visualization of viewing fields in section \ref{VisualizationRequirements}, then the detailed definitions of the visual aids are given in section \ref{VisualizationMethods}.

%------------------------------------------------------------------------------

\section{Visualization Elements}
\label{VisualizationRequirements}

Viewing volume of a projective camera is defined by an infinite half cone, one apex of which corresponds to the focal point of the camera. Because real computers cannot have unlimited resources, in practical computer 3D technology we usually limit the bottom of the cone by a far clip plane or by boundary surfaces which may cut the viewing volume, then limit the top of the cone even more by a near clip plane. This results in a frustum (figure \ref{fig:ViewingVolume}). In this research we abstract the viewing volume for visualization by the term ``viewing field''.

\begin{figure}[htbp]
	\centering
	\includegraphics{./Primitives/viewing_volume.png}
	\rule{35em}{0.5pt}
	\caption[Viewing volume of camera]{Viewing volume of camera}
	\label{fig:ViewingVolume}
\end{figure}

There may be many methods for visualizing the viewing fields, not limited to the ones discussed in the next section. But for a method to find practical use in outdoor MR, it should be able to be implemented (see section \ref{UseCase} to have an illustration) so that it meets the following basic qualitative requirements, expressed in the ``it should'' tongue of most Domain Specific Languages of the Behavior Driven Development methodology in software engineering:

\begin{itemize}
	\item It should be easy to understand when users see the visualized viewing fields of surveillance cameras.
	\item It should work in realtime. When users change the position or orientation of the mobile device, the MR video displayed on the screen of the mobile device should simultaneously change accordingly to the movement of the camera with no or little delay.
	\item It should have good accuracy. In order to correctly overlay visual aids onto the original video frames taken at the user side, the position and orientation of the mobile camera must be estimated within a small registration error.
	\item It should be robust to disturbance in outdoor environment. Some of the disturbance that may arise in real situations are passers in front of the mobile camera and GPS signal noise/weakness. For example, in outdoor environment, there may be bicycle riders passing by and they may temporarily occlude the mobile camera. Another example is that if the system uses GPS device, the GPS signal strength may change a lot when users walk from an open space to a space shadowed by trees or buildings.
\end{itemize}

We must take the above requirements into account when implementing the prototype in chapter \ref{Chapter3}.

%------------------------------------------------------------------------------

\section{Visualization Methods}
\label{VisualizationMethods}

In all the five visualization methods, in order to emphasize the boundary of the viewing fields of the surveillance camera, we draw four straight lines to visualize the edges of the cone in figure \ref{fig:ViewingVolume}. The way we draw the five sides and the inside space of the cone gives us various visualization methods. A good side effect when drawing the four straight lines is that the position of the camera can be inferred because it is the only intersection point of the lines.

\subsection{Volume Method}

The most direct method to visualize the viewing field is simply visualizing the viewing volume (figure \ref{fig:VolumeMethod}) in alpha blending (see-through) fashion. However, the volume usually occupies parts of the mobile device's screen, especially if the user is inside the viewing volume, hence it may be hard to be understood.

\begin{figure}[htbp]
	\centering
	\includegraphics[width=14cm]{./Primitives/theory_volume.png}
	\rule{35em}{0.5pt}
	\caption[Volume method]{The viewing field of a surveillance camera on the building is being visualized using the volume method. The user's position is indicated by the green dot.}
	\label{fig:VolumeMethod}
\end{figure}

\subsection{Shadow Method}
\label{ShadowMethod}

The fact that the frame captured by a surveillance camera can be thought to be a perspective projection of the volume into the near plane of the frustum gives us another method, shadow method. The shadow casts the virtual shadow created by the light positioned at the focal-point of the camera and the near plane of the frustum (figure \ref{fig:ShadowMethod}). This method may give more understandable visualization.

\begin{figure}[htbp]
	\centering
	\includegraphics[width=14cm]{./Primitives/theory_shadow.png}
	\rule{35em}{0.5pt}
	\caption[Shadow method]{The viewing field of a surveillance camera on the building is being visualized using the shadow method. The user's position is indicated by the green dot.}
	\label{fig:ShadowMethod}
\end{figure}

There are various ways to render the shadow, the two popular ones are shadow mapping \cite{Reference7} \cite{Reference8} and volume shadow \cite{Reference9}.

\subsection{Contour Method}

An alternative method is to only visualize the contours of the shadow (figure \ref{fig:ContourMethod}) to reduce the occlusion of the scene by the visual aid. Rendering only the contour is much lighter than rendering the whole shadow, thus this method is supposed to be much faster than the shadow method.

\begin{figure}[htbp]
	\centering
	\includegraphics[width=14cm]{./Primitives/theory_contour.png}
	\rule{35em}{0.5pt}
	\caption[Contour method]{The viewing field of a surveillance camera on the building is being visualized using the contour method. The user's position is indicated by the green dot.}
	\label{fig:ContourMethod}
\end{figure}

\subsection{Arrow Method}

In some cases, a person may want to know the relative distances from points inside the shadow (section \ref{ShadowMethod}) to the camera, in addition to its viewing field itself. To visualize this information, we propose vector method that puts arrows on the shadow area as in figure \ref{fig:ArrowMethod}. In the figure, all arrows are pointing the camera, and their lengths are proportional to the distance from the root of the arrows to the camera.

\begin{figure}[htbp]
	\centering
	\includegraphics[width=14cm]{./Primitives/theory_arrow.png}
	\rule{35em}{0.5pt}
	\caption[Arrow method]{The viewing field of a surveillance camera on a building is being visualized using the arrow method. The arrows' roots are indicated by blue dots, the arrows' heads are not drawn for clarity. The user's position is indicated by the green dot.}
	\label{fig:ArrowMethod}
\end{figure}

\subsection{Animation Method}

This method uses a moving mesh (figure \ref{fig:AnimationMethod}) coming from the surveillance camera to the clipped end of the viewing field (or in the reverse direction) to visualize the viewing fields. This method has  many advantages:

\begin{itemize}
	\item The moving mesh starts from the surveillance camera position (or moves towards the camera). As a result the user can easily know the camera position.
	\item Even when the mobile device is fixed at a certain position and pose, the visualization effect is still helpful to understand the viewing field because of its animation.
	\item The moving mesh does not occlude the scene. As a result the user can easily see both the scene and the visualized viewing field at the same time.
\end{itemize}

\begin{figure}[htbp]
	\centering
	\includegraphics[width=14cm]{./Primitives/theory_animation.png}
	\rule{35em}{0.5pt}
	\caption[Animation method]{The viewing field of a surveillance camera on the building is being visualized using the animation method. The mesh is moving from the camera down to the ground. The user's position is indicated by the green dot.}
	\label{fig:AnimationMethod}
\end{figure}

To make the visual aids look more ``real'', the moving mesh stops when it touch the surface of the building walls or the ground. In order to do this, we need the model of the scene as described in the next section.

%------------------------------------------------------------------------------
\section{Scene Model}
\label{SceneModel}

In order to define the viewing field for the visualization methods in section \ref{VisualizationMethods}, we need the model of the scene. The model contains the 3D structure of the buildings (figure \ref{fig:SceneModel}) and the positions and orientations of the surveillance cameras. The model is rather simple and only contains flat planes. We do not need to have complexed ones, such as those that contain texture, because we only need a coarse-grained approximation of the viewing field. Moreover, because surveillance cameras are usually stationary, their positions and orientations are usually static.

\begin{figure}[htbp]
	\centering
	\includegraphics[width=14cm]{./Primitives/scene_model.png}
	\rule{35em}{0.5pt}
	\caption[Scene model]{3D structure of the buildings in the scene model}
	\label{fig:SceneModel}
\end{figure}

Using the scene model, we can also render visual aids in various modes as described in the next section.

%------------------------------------------------------------------------------

\section{Viewing Modes}
\label{ViewingModes}

In general it is easier to understand a thing when it is put in a context. In our case, in real scene when users see our visual aids that visualize the viewing fields of the cameras, they also see the big picture of the area around them. Consequently, if we let the users see the map of the area where they are standing, the map will help them understand the visualized viewing fields of the surrounding surveillance cameras better.

Because we have the 3D scene model as described in the previous section, we propose three viewing modes:

\begin{itemize}
	\item Real mode: In this mode frames taken by the mobile camera are overlaid with any of the visual aids.
	\item Map mode: Map of the area around the current position is displayed to let users overview the scene.
	\item First-person shooter (FPS) mode: Users can navigate the 3D world as in FPS games. Using this mode, users can virtually walk to other positions of the area to see other viewing fields of surveillance cameras there. In this mode only the scene model is rendered, video frames taken by the mobile camera are not used.
\end{itemize}

In section \ref{VisualizationMethods}, figures on the left and on the right show the screen in real mode and map mode respectively.

In order to navigate the scene in map and FPS mode, we can use the keypads on the mobile devices. However, keypads on mobile devices are usually hard to use, especially when only one hand is used to both hold the device and press the keys. When a mobile device does not have a touch screen, but it is equipped with an accelerometer (most cell phones today have one built-in) or gyrocompass, these auxiliary devices can be used for navigating around. This usage has been proved by existing applications of some game gadgets and iPhone that have those sensors. For example, in map mode the three degree-of-freedom parameters of the gyrocompass can be used as in table \ref{tb:GyrocompassNavigation}.

\begin{table}[tb]
	\begin{center}
		\caption{Using gyrocompass for navigation}
		\label{tb:GyrocompassNavigation}
		\begin{tabular}{|c|l|l|}
			\hline
			Degree-of-freedom parameter & Usage                     \\
			\hline
			Yaw                         & 360$^\circ$ rotation      \\
			Pitch                       & Forward-backward movement \\
			Roll                        & In/out zooming            \\
			\hline
		\end{tabular}
	\end{center}
\end{table}

\begin{figure}[htbp]
	\centering
	\includegraphics{./Primitives/yaw_pitch_roll.png}
	\rule{35em}{0.5pt}
	\caption[The three degree-of-freedom parameters of a gyrocompass]{The three degree-of-freedom parameters of a gyrocompass}
	\label{fig:YawPitchRoll}
\end{figure}

For devices that have multi-touch screen like the iPhone, the touch screen can be used to move, rotate, zoom in and out the map in the map viewing mode very easily.

\chapter{Visualization Methods}
\label{Chapter3}
\lhead{Chapter 3. \emph{Visualization Methods}}

\section{Volume method}

\section{Shadow method}

\section{Silhouette method}

\section{Arrow method}

\section{Animation method}

\chapter{Prototype Implementation}
\label{Chapter4}
\lhead{Chapter 4. \emph{Prototype Implementation}}

In this chapter, we implement a prototype based on PTAM \citep{Reference12}. The prototype integrates the five methods in section \ref{VisualizationMethods}. A use case of this system is in section \ref{AUseCase}.

\section{System Architecture}

The mobile device used in the prototype is a SONY VAIO VGN-UX90PS (see appendix \ref{AppendixA}). We base our program on the PTAM reference implementation source code \citep{Reference16} to get the position and orientation of the VAIO. However, this implementation cannot run on the VAIO because it requires a dual core CPU computer to be able to run smoothly. As a result, we connect a MacBook Pro (appendix \ref{AppendixB}) to the VAIO and the PTAM part of the program will run on the MacBook. The architecture is in figure \ref{VAIOMacBookPro}.

The VAIO and the MacBook are connected by a 11 Mbps wireless network. For the sake of quick setup time, the VAIO and the MacBook are connected directly using ad hoc mode, which means that an additional wireless hub is not needed.

The MacBook may run at 54 Mbps, but the VAIO's built-in wireless network card can only run at 11 Mbps. The 11 Mbps is sufficient for the current prototype. For faster network speed, we can attach a 54 Mbps auxiliary wireless network card for the VAIO. We can also use LAN cable for even faster 100 Mbps speed, but the VAIO will be wired to another bulky device, which is not desirable because the mobility of the VAIO is totally lost.

\section{Server-side PTAM}

We modify the original PTAM source code so that the ``video source'' is remotely located on the VAIO, not on the MacBook. We learn a little from Erlang language \citep{Reference17} of how to write concurrent, high-performance systems. Right after the TCP (Transmission Control Protocol) connection between the VAIO and the MackBook is established:

\begin{itemize}
	\item The VAIO continuously feeds frames captured from its camera to the MacBook.
	\item The MacBook continuously feeds position and orientation of the camera of the VAIO to the VAIO.
\end{itemize}

As the above explanation suggests, we use ``push'' instead of ``pull'', one side actively sends data to the other side without having to wait for the request for the data from the other side. In other words we use the mailbox design pattern instead of the RPC (Remote Procedure Call) design pattern, which is slower and requires synchronizing the call and the result (figure \ref{fig:MailboxRPC}).

\begin{figure}[htbp]
	\centering
	\includegraphics{./Figures/mailbox_rpc.png}
	\rule{35em}{0.5pt}
	\caption[Mailbox vs. RPC]{Mailbox (above) vs. RPC (below)}
	\label{fig:MailboxRPC}
\end{figure}

Because PTAM only need grayscale frames, instead of sending all the three RGB color channels of the frames, the VAIO only need to send the brightness Y = 0.3*R + 0.59*G + 0.11*B. To increase the frame rate sent to the MacBook, we further improve the data transmission by the following optimizations:

\begin{itemize}
	\item Send G color channel instead of Y, because G affects Y the most as in the above common equation. This does not affect PTAM. This optimization enormously improves the frame rate because when we send Y, if the image size is 320x240, we must apply the above equation 76800 times for each frame!
	\item Losslessly compress G color channel using the standard library ``zlib''. The compression rate is about 50\%.
\end{itemize}

\begin{table}[tb]
	\begin{center}
		\begin{tabular}{|c|l|l|}
			\hline
			Sent color data & FPS \\
			\hline
			Y               &     \\
			G               &     \\
			compressed G    &     \\
			\hline
		\end{tabular}
		\caption{Frame rate improves when applying frame transmission optimizations}
	\end{center}
\end{table}

The prototype uses frame size of 320x240. This gives smooth frame rate with be above system architecture and optimizations. Bigger frame size allows PTAM to find more feature points, but the following drawbacks pop up:

\begin{itemize}
	\item Data for each frame gets bigger. This increases data transmission time for each frame.
	\item The MacBook must process more data. This decreases the result frame rate.
\end{itemize}

When increasing from 320x240 to 640x480, the frame rate drop about 4 times, and the prototype no longer works smoothly as required in \ref{VisualizationRequirements}. 

\section{PTAM's Map Initializing Problem}

PTAM continuously build and maintain a map of feature points taken from the frames.

\section{Pre-registered Data}

The system does not use a server or any real surveillance camera. In place of those, 3D model of the building of the College of International Studies and the gymnasium (Figure [cite]), together with the hard coded geometry information about the virtual surveillance cameras are used.

\section{History of this prototype}

The prototype described in the above sections is not the only decision choice we have made throughout this research. Previously, we have made others as described below.

\subsection{Hardware -- Prototype with SONY VAIO VGN-UX90PS, Garmin GPSmap 60CSx, and InterSense InertiaCube3}

We have implemented a prototype with the architecture in figure \ref{fig:VAIOGPSGyro}. The GPS device specifications is in appendix \ref{AppendixD}. The GSP and gyrocompass devices provide position and orientation information of the mobile device. Together with the prebuilt CG information, we can overlay CG objects to visualize viewing fields of surveillance cameras. The update rate of the gyrocompass is 180Hz, which is high thus the orientation information is quite good. However, the GPS device error is about 10m, which affects badly the accuracy of the visualization. To improve the GPS device error, model-based tracking method as described in \citep{Reference13} method may be used.

\begin{figure}[htbp]
	\centering
	\includegraphics{./Primitives/vaio_gps_gyro.png}
	\rule{35em}{0.5pt}
	\caption[Prototype with SONY VAIO VGN-UX90PS, Garmin GPSmap 60CSx, and InterSense InertiaCube3]{Prototype with SONY VAIO VGN-UX90PS, Garmin GPSmap 60CSx, and InterSense InertiaCube3}
	\label{fig:VAIOGPSGyro}
\end{figure}

In the above prototype, we do not use GPS and gyrocompass devices. But GPS and gyrocompass devices can be used together with PTAM in the multi-sensor fusion \citep{Reference14} style, which may greatly improves the robustness of the system:

\begin{itemize}
	\item Initialization problem: the GPS and gyrocompass devices may provide initial position and orientation of the mobile device. Although the error of the GPS device is not small, in a large map it may help PTAM to reduce the search space.
	\item Quick movement: PTAM is image-based, thus it does not work well when the user quickly moves the mobile device. At this time, the gyrocompass may provide PTAM the orientation information because its update rate is high. Although the gyrocompass suffers from drifting error and needs to be continuously adjusted by the other sensors in the fusion, quick mobile device movement usually lasts only in a short time, thus gyrocompass may provide precious temporary information during this time.
\end{itemize}

\subsection{Software -- Development environment}

Visualization methods are expected to be found experimentally and visually. Thus, trial and error methodology is applied here. To shorten the trial and error cycle, we need a good development environment which allows quick compiling, running and modifying.

At first we used C/C++ language and OpenGL API \citep{Reference10}. Because both the language and the API are in too low level, the development speed was slow. Later, C/C++ language was fully replaced by Ruby language. Ruby is an object oriented strongly-typed dynamic language that attracts attention of developers world-wide in recent years. From the perspective of software engineering, the above features provide the following benefits:

\begin{itemize}
	\item Object oriented feature: Easy to structure the program and later maintain the source code over time.
	\item Strongly-typed feature: Variables have types, thus we can avoid bugs which usually occur in weakly-typed languages like PHP.
	\item Dynamic feature: There is no compile time. With static languages like C/C++, a lot time is wasted on compiling source code. With Ruby we can modify source code and run right away.
\end{itemize}

The development speed was better but still slow, partly because of the slow. It was concluded that the speed of the development is largely affected by the API rather than the language. Consequently, a higher level 3D rendering engine has been adopted: Irrlicht. Later, we read many reviews on the Internet that say that OGRE \citep{Reference11} is easier to use than Irrlicht, thus we migrated to OGRE. OGRE, Object-Oriented Graphics Rendering Engine, is a scene-oriented, flexible 3D rendering engine written in C++ designed to make it easier and more intuitive for developers to produce applications utilizing hardware-accelerated 3D graphics. The class library abstracts all the details of using the underlying system libraries like Direct3D and OpenGL and provides an interface based on world objects and other intuitive classes.

However, OGRE is generally used for creating 3D games thus contain too many features that we do not need. The framework forced us to write too much bloated code for the purpose of this research. As a result, we finally migrated to a combination of Ruby and C language with pure OpenGL API. At this time Ruby has grown to version 1.9. In this version the interpreter is replaced by a virtual machine, which boosts our Ruby program's speed to about 10x. Moreover importantly it is ridiculously easy to write C extension for Ruby \citep{Reference15}. For program parts that need speed like networking and image processing, we use the Ruby standard library which is written in C, or write them as C extension for Ruby. For program parts that need trial-error or does not need speed like the user interface part, we write them in pure Ruby.

In short, our experience suggests that keeping a balance between dynamic language and static language is a good choice.

\chapter{Evaluation of Visualization Methods}
\label{Chapter5}
\lhead{Chapter 5. \emph{Evaluation of Visualization Methods}}

The visualization methods as described in chapter \ref{Chapter3} needs to be evaluated to check for understandability and to see which one is the best one users. In this chapter, we use the reference implementation built in chapter \ref{Chapter4} to setup an experiment in which users are invited to use the system with the five visualization methods. After that, the users are asked to rank the understandability of each method in five levels, from the least to the most understandability.

\section{Evaluation condition}

...

\section{Result and discussion}

...


\chapter*{Acknowledgements}
\addcontentsline{toc}{chapter}{\numberline{}Acknowledgements}

\newpage

\appendix % Cue to tell LaTeX that the following 'chapters' are Appendices

\chapter{Sony VAIO VGN-UX90PS}
\label{AppendixA}

See figure \ref{fig:VAIO} and \ref{fig:VAIOBack}.

\begin{itemize}
	\item Size: 150.2 x 95 x 32.2 mm
	\item Weight: 520 g
	\item CPU: Core solo U1400 1.20 GHz
	\item RAM: 512 MB DDR2
	\item OS: Microsoft Windows XP Professional (SP2)
	\item LCD Display: 4.5 inch wide TFT color LCD
	\item Display mode: 16190000 colors
	\item Video memory: 128 MB
	\item Interface:
		\begin{itemize}
			\item Hi-speed USB (USB 2.0)
			\item Network (LAN) connector
			\item Wireless IEEE801.11a/b/g, WPA2. Wi-Fi. Bluethooth2.0
		\end{itemize}
	\item Camera: Web camera MOTION EYE x 2
		\begin{itemize}
			\item 1/8 inch VGA Progressive CMOS, 3.3 M pixel, f = 2.6 mm F4
			\item 1/4 inch SXGA Progressive CMOS, 13.4 M pixel, f = 3.8 mm F4
		\end{itemize}
\end{itemize}

\chapter{Sony Vaio VGN-UX90PS}
\label{AppendixB}
\lhead{Appendix B. \emph{Sony Vaio VGN-UX90PS}}

Write your Appendix content here.

\chapter{InterSense InertiaCube3}
\label{AppendixC}
\lhead{Appendix C. \emph{InterSense InertiaCube3}}

InterSense InertiaCube3 specifications.

\begin{itemize}
\item Size: 26.2mm x 39.2mm x 14.8mm
\item Update rate: 180Hz
\end{itemize}

\input{./Appendices/AppendixD}

%% Bibliography
\addcontentsline{toc}{chapter}{\numberline{}Bibliography}
\renewcommand{\bibname}{Bibliography}
% use bibtex
\bibliographystyle{unsrt}
%\bibliography{samplebib}
%% [compile] bibtex sample-en; platex sample-en; platex sample-en;

%\bibliographystyle{unsrtnat}  % Use the "unsrtnat" BibTeX style for formatting the Bibliography
\bibliography{Bibliography}  % The references (bibliography) information are stored in the file named "Bibliography.bib"

\end{document}
